\section{WiFi Direct}

La definición de WiFi Direct por la WiFi Alliance, organización creadora de esta tecnología, como una "certificación para diferentes dispositivos que soportan cierta tecnología que permite la comunucación directa", en otras palabras, permite conectar distintos dispositivos sin necesidad de cables. Esta tecnología es en la que se basa el protocolo Miracast.

WiFi Direct es en esencia un punto se acceso en forma de software, los llamados 'Soft AP'. El Soft AP proporciona una versión de Wi-Fi Protected Setup o 'WPS'. Respecto a la seguridad, Wi-Fi Direct incluye seguridad WPA2 y ofrece controlar el acceso a redes corporativas. Los dispositivos certificados para Wi-Fi Direct se pueden conectar "uno a uno" o "uno a muchos", y no todos estos dispositivos conectados necesitan tener Wi-Fi Direct, con solo un dispositivo Wi-Fi Direct habilitado se pueden conectar los demás dispositivos con el estándar previo de Wi-Fi. Esta tecnología se puede ver claramente, por ejemplo, cuando se comparte internet con otro teléfono móvil.

El funcionamiento de WiFi Direct es similar al de Bluetooth. Un dispositivo con  WiFi Direct activado emite una señal hacia otros dispositivos haciendo saber que está disponible para una conexión. Los usuarios pueden enviar una petición, o recibir una, para efectuar dicha conexión. Cuando ambos o más dispositivos están conectados se puede empezar a compartir archivos.

La principal diferencia con Bluetooth es la mayor tasa de transferencia de archivos, 250Mbps de WiFi Direct respecto a los 25Mbps de Bluetooth 4.0, aunque los dos hagan uso del protocolo 802.11.

En definitiva, Wi-Fi Direct, es una versión actualizada y mejorada de Bluetooth sin necesidad de tener que establecer conexión a Internet.
