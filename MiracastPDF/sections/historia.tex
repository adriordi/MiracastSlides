\section{Historia}
\addcontentsline{toc}{section}{Historia}
Miracast fue anunciado en 2013 en el congreso tecnologico de las vegas conocido como CES por la Wifi Alliance, era un protocolo revolucionario que permitia compartir contenidos multimedia inalambricamente, al igual que hasta el momento se podía hacer con un cable hdmi o VGA con la inestimable ventaja de poder prescindir de los cables, ya que todo funciona inalambricamente.

La certificacion miracast tuvo un gran calado en la industria de consumo multimédia, y en pocos meses todos los grandes de la electronica de consumo anunciaron nuevos productos compatibles con esta tecnología, como televisiones, moviles, etc.

Aunque no seria hasta octubre del año siguiente y los meses que lo seguirían que esta tecnología viviría su mayor empuja, gracias a la competencia Apple vs Google. La primera había anunciado Airplay, un estandar similar a miracast, y Google en su intento por no quedar atras en esa carrera tecnológica añadio miracast al código fuente de android, facilitando asi que todos los fabricantes de su ecosistema pudiesen implementar facilmente esta tecnología, con lo que se produjo una gran expansión de dispositivos compatibles.
El segundo gran empujon llego con la presentación de chromecast, un dongle HDMI que conectandose al puerto HDMI de una pantalla y compartiendo el mismo wifi que un pc o un movil, era capaz de hacer mirroring en la pantalla, con las grandes aplicaciones que esto presenta.
A dia de hoy una gran cantidad de disposítivos y aplicaciones hacen uso de Miracast para ampliar su utilidad y seguir facilitando la vida a los usuarios.
