\section{CODECs}

Miracast dispone de un códec obligatorio y dos opcionales.

El códec obligatorio es LPCM, 16 bits 48 KHz en 2 canales. LPCM son las siglas de Linear Pulse Code Modulation, es un formato de audio sin compresión que puede llegar a tener hasta 8 canales de audio en una frecuencia que va desde los 48 KHz hasta los 96 KHz. Es muy usado en comunicaciones y ahora también ha sido adoptado por la industria musical. El formato simultáneamente captura y muestrea señales analógicas y las transforma en señales digitales. El PCM Lineal, conocido así en España, fue definido como parte del estándar DVD.

En los códecs opcionales tenemos, AAC (del inglés Advanced Audio Coding) es un formato de señal digital de audio basado en un algoritmo de compresión con pérdida, un proceso por el que se eliminan algunos datos de audio para poder obtener el mayor grado de compresión posible, resultando en un archivo de salida que suena lo más parecido posible al original, éste algoritmo de codificación de banda ancha de audio tiene un rendimiento superior al del MP3, mejor calidad en archivos más pequeños y requiere menos recursos del sistema para codificar y decodificar. El formato AAC corresponde al estándar internacional "ISO/IEC 13818-7" como una extensión de MPEG-2 (Moving Picture Experts Group). Este formato ha sido elegido por Apple como formato principal para los iPods y para su software iTunes. Entre las extensiones de archivo de este formato cabe destacar .m4a, .3gp, .mp4, entre otras.

Por último tenemos el códec AC-3, conocido como Dolby Digital. Es una serie de tecnologías de compresión de audio desarrollado por los laboratorios Dolby. El AC-3 es uno de los formatos denominados de compresión perceptual. Lo que hace, básicamente, es eliminar todas las partes del sonido original, codificado analógicamente, que no pueda ser percibido por el oído humano. De ésta forma, se logra que la misma información sea de menor tamaño y por lo tanto ocupe mucho menos espacio físico. AC-3, es la versión más común que contiene hasta un total de 6 canales de audio, con 5 canales de ancho de banda completo de 20 Hz - 24 KHz para los altavoces de rango normal y un canal de salida exclusivo para los sonidos de baja frecuencia conocida como Low Frequency Effect, o subwoofer. También soporta el uso de Mono y Stereo.
