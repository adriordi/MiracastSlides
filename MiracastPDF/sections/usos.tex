\section{Usos y aplicaciones}

Ya se han ejemplificado varios escenarios en los que la tecnología Miracast puede resultar muy práctica. Siendo muy usada por ejemplo para mostrar datos y presentaciones en reuniones empresariales, ya que permite conectar el pc a un proyector compatible en pocos segundos y empezar, sin cables de por medio. También es muy usado en el ámbito informático, ya que gracias al mirroring se pueden mostrar demos muy fácilmente en cualquier momento.

En el ámbito doméstico, da facilidades para compartir contenido con un gran número de personas mediante una televisión compatible, como mostrar las fotos de las vacaciones en la tele, reproducir música por los altavoces de una fiesta, o incluse realizar videollamadas haciendo uso de la televisión, aunque todo esto requiere tener el dispositivo emisor siempre encendido.

Por otro lado la tecnología que usa por debajo, WiFi Direct es muy usada para intercambio de archivos, se habla de que incluso podría reemplazar al Bluetooht, ya que puede conectar a varios dispositivos en una LAN sin necesidad de router, y gracias a su funcionamiento P2P el crecimiento que podría experimentar esta LAN es enorme, ya que los nuevos dispositivos se conectan a otros que ya hay conectados, no al que se estableció como punto de acceso original.

Todo esto se vera con más detalle cuando estudiemos los apartados técnicos del protocolo.
