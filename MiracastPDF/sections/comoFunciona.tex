\section{¿Cómo funciona?}

El funcionamiento de Miracast es sencillo, conectar todos los dispositivos sin necesidad de Internet, solo necesitan estar conectados a una red local. Tanto el dispositvo emisor como el receptor deben soportar la tecnología Miracast para funcionar. Sin embargo, para transmitir contenido multimedia a un dispositivo que no es compatible con Miracast, existen adaptadores que se conectan a los puertos HDMI o USB.

Miracast permite a un dispositivo portátil, ya sea móvil o tablet, o a un ordenador enviar, de forma segura, vídeo de alta definición hasta 1080p y sonido envolvente 5.1. Permite a los usuarios, por ejemplo, duplicar la pantalla de sus smartphones en un televisor, e incluso compartir la pantalla de un ordenador portátil con el proyector en una sala de coferencias en tiempo real, para que todos los asistentes puedan ver, por ejemplo una presentación.
