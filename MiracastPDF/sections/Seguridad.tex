\sections{Seguridad}

Respecto a la seguridad, Miracast cuenta con el protocolo de seguridad WPA2.

WPA2 es el nuevo estándar del IEEE (Institute of Electrical and Electronics Engineers) para proporcionar seguridad en redes WLAN. WPA2 incluye el algoritmo de cifrado AES, desarrollado por el NIST (National Institute of Standards and Technology). Se trata de un algoritmo de cifrado en bloque con claves de 128 bits.

WPA2-PSK soporta una contraseña de hasta 63 caracteres alfanúmericos, es decir que la contraseña puede llegar a tener hasta 63 mayúsculas, minúsculas, símbolos y números, por lo que la clave se hace robusta para su descifrado. Además, a partir de la Pre-Shared Key, el sistema va generando nuevas claves que transmite al resto de equipos, lo cual dificulta notablemente la acción de descifrado. Un incoveniente importante es que todos los dispositivos no soportan el modo WPA2-PSK.

En definitiva, respecto a seguridad Miracast cuenta con un gran sistema de protección que dificultad bastante que personas ajenas puedan llegar a visualizar lo que estamos compartiendo a través de este sistema.
