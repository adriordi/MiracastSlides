%%%%%%%%%%%%%%%%%%%%%%%%%%%%%%%%%%%%%%%%%
% University/School Laboratory Report
% LaTeX Template
% Version 3.1 (25/3/14)
%
% This template has been downloaded from:
% http://www.LaTeXTemplates.com
%
% Original author:
% Linux and Unix Users Group at Virginia Tech Wiki 
% (https://vtluug.org/wiki/Example_LaTeX_chem_lab_report)
%
% License:
% CC BY-NC-SA 3.0 (http://creativecommons.org/licenses/by-nc-sa/3.0/)
%
%%%%%%%%%%%%%%%%%%%%%%%%%%%%%%%%%%%%%%%%%

%----------------------------------------------------------------------------------------
%	PACKAGES AND DOCUMENT CONFIGURATIONS
%----------------------------------------------------------------------------------------

\documentclass{article}

\usepackage[version=3]{mhchem} % Package for chemical equation typesetting
\usepackage{siunitx} % Provides the \SI{}{} and \si{} command for typesetting SI units
\usepackage{graphicx} % Required for the inclusion of images
\usepackage{natbib} % Required to change bibliography style to

% \setlength\parindent{0pt} % Removes all indentation from paragraphs

\renewcommand{\labelenumi}{\alph{enumi}.} % Make numbering in the enumerate environment by letter rather than number (e.g. section 6)

%\usepackage{times} % Uncomment to use the Times New Roman font

% Para usar el español sin demasiadas complicaciones

\usepackage[spanish]{babel}
\selectlanguage{spanish}
\usepackage[utf8]{inputenc}
%----------------------------------------------------------------------------------------
%	DOCUMENT INFORMATION
%----------------------------------------------------------------------------------------

\title{Analisis y estudio del protocolo Miracast} % Title

\author{Adrian Orduña Diaz, Rafael Leyva Ruiz \\ Grupo 13} % Author name

\date{\today} % Date for the report

\begin{document}

\maketitle % Insert the title, author and date


% If you wish to include an abstract, uncomment the lines below
% \begin{abstract}
% Abstract text
% \end{abstract}

%----------------------------------------------------------------------------------------
%	Introduccion
%----------------------------------------------------------------------------------------

% \section{Introducción}

Miracast es definido en la web de \href{http://www.wi-fi.org/discover-wi-fi/wi-fi-certified-miracast}{WiFi Alliance} más como una certificación más que como un protocolo, ya que para su funcionamiento se basa en diversos protocolos que trabajan todos juntos, pero igualmente se puede analizar desde el punto de vista de considerarlo un protocolo en sí, debido a la gran cantidad de requerimientos técnicos que conlleva y su forma de trabajar, que propiamente define un protocolo de conexión, aunque el, llamemoslo asi, trabajo sucio lo realizan otros protocolos. Por eso este documento busca dar una introducción y base de conocimientos acerca de Miracast, su funcionamiento desde un punto de vista técnico, utilidades y beneficios a nivel de usuario.


\section{Historia}

Miracast fue anunciado en 2013 en el congreso tecnologico de las vegas conocido como CES por la Wifi Alliance, era un protocolo revolucionario que permitia compartir contenidos multimedia inalambricamente, al igual que hasta el momento se podía hacer con un cable hdmi o VGA con la inestimable ventaja de poder prescindir de los cables, ya que todo funciona inalambricamente.
\\
La certificacion miracast tuvo un gran calado en la industria de consumo multimédia, y en pocos meses todos los grandes de la electronica de consumo anunciaron nuevos productos compatibles con esta tecnología, como televisiones, moviles, etc.
\\
Aunque no seria hasta octubre del año siguiente y los meses que lo seguirían que esta tecnología viviría su mayor empuja, gracias a la competencia Apple vs Google. La primera había anunciado Airplay, un estandar similar a miracast, y Google en su intento por no quedar atras en esa carrera tecnológica añadio miracast al código fuente de android, facilitando asi que todos los fabricantes de su ecosistema pudiesen implementar facilmente esta tecnología, con lo que se produjo una gran expansión de dispositivos compatibles.
\\
El segundo gran empujon llego con la presentación de chromecast, un dongle HDMI que conectandose al puerto HDMI de una pantalla y compartiendo el mismo wifi que un pc o un movil, era capaz de hacer mirroring en la pantalla, con las grandes aplicaciones que esto presenta.
\\
A dia de hoy una gran cantidad de disposítivos y aplicaciones hacen uso de Miracast para ampliar su utilidad y seguir facilitando la vida a los usuarios.


\end{document}
